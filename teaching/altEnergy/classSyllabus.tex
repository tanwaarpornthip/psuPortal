\documentclass[a4paper,12pt]{article}

% Set the locale for linebreak to Thai
%\XeTeXlinebreaklocale "th"
% In English, when TeX tries to justify text,
% it will add some spaces between words
% For Thai, we "must not" add any space between words
% i.e. put "zero" space beteween words
%\XeTeXlinebreakskip = 0pt plus 0pt
% For a bit better(?) justified output
%\sloppy

% For any unicode characters, require XeTeX/XeLaTeX
\usepackage{fontspec}
\usepackage{subcaption}
\usepackage[document]{ragged2e}
\defaultfontfeatures{Mapping=tex-text} 

% Set main fonts
% For Thai, I recommend to scale the size to the uppercase size of latin alphabet
%\setmainfont[Scale=MatchUppercase,Mapping=tex-text]{TH Sarabun New}
\setmainfont{TeX Gyre Termes}				% Free Times

% Sans font
\setsansfont{TeX Gyre Heros}				% Free Helvetica

% Monospace font
\setmonofont{TeX Gyre Cursor}				% Free Courier
%%%%%%%%%%%%%%%%%%%%%%%%%%%%%%%%%%%%%%%%%%%%%%%%%%%%%%%%%%%%%%%%%%%%%%%%%
\usepackage[top=2cm, bottom=2cm, left=2cm, right=2cm]{geometry}
%\addto{\captionsthai}{\renewcommand{\abstractname}{รายละเอียดของโครงการโดยย่อ}}
\renewcommand{\abstractname}{Project Overview}

% For changing enum label types
\usepackage{enumitem}


\usepackage{listings}
\lstset{
    language=TeX,
    basicstyle=\ttfamily,
    numbers=left,
    numberstyle=\small,
    breaklines=true,
    xleftmargin=.05\linewidth,
    frame=single,
    columns=fullflexible,
    captionpos=b,
    showstringspaces=false,
}



% Any percent sign marks a comment to the end of the line

% Every latex document starts with a documentclass declaration like this
% The option dvips allows for graphics, 12pt is the font size, and article
%   is the style

\usepackage{graphicx}
\usepackage{url}

% These are additional packages for "pdflatex", graphics, and to include
% hyperlinks inside a document.

%\setlength{\oddsidemargin}{0.25in}
%\setlength{\textwidth}{6.5in}
%\setlength{\topmargin}{0in}
%\setlength{\textheight}{8.5in}

\usepackage{authblk}
% Use thai separation for Thai document
%\renewcommand\Authand{ และ }

% IndentFirst line
\usepackage{indentfirst}

% Using hyperlink in document
\usepackage{hyperref}

% Using cleverref for referencing items in the document
\usepackage{cleveref}

% Wrap figure use
\usepackage{wrapfig}

% Table accommodation
%\usepackage{array}
\newcommand*{\thead}[1]{\textbf{#1}}

%Set unnumbered section
\setcounter{secnumdepth}{0}

%Paragraph indentation settings
\setlength{\RaggedRightParindent}{2em}

\begin{document}

% Everything after this becomes content
% Replace the text between curly brackets with your own

\title{Course syllabus\\ \large 988-342 Alternative Energy \& Energy conservation}
\author{Dr. Tanwa Arpornthip}
\date{}

% You can leave out "date" and it will be added automatically for today
% You can change the "\today" date to any text you like
% A blank date can be given if date is not desired in the title

\maketitle
% This command causes the title to be created in the document

\noindent\makebox[\linewidth]{\rule{\linewidth}{0.4pt}}
\begin{description}
\item [Office] Building 6, ESSAND faculty room
\item [Office Hours] \hfill \\
    Monday 14:00-14:50 ESSAND meeting room \hfill\\
    Friday 12:00-14:00 ESSAND meeting room
\item [Email address] \href{mailto:tanwa.a@phuket.psu.ac.th}{tanwa.a@phuket.psu.ac.th}
\item [Work phone] 076-276-435
\item [Web page] \href{http://www.essand.psu.ac.th/ESSAND/Tanwa}{http://www.essand.psu.ac.th/ESSAND/Tanwa}
\item [Course page] \href{http://www.essand.psu.ac.th/ESSAND/Tanwa/altEnergy}{http://www.essand.psu.ac.th/ESSAND/Tanwa/altEnergy}
\item [Class hours] Monday, Wednesday, Friday 16:30-17:30 Room 5407A
\end{description}
\noindent\makebox[\linewidth]{\rule{\linewidth}{0.4pt}}

\begin{enumerate}[label=\textbf{\Alph*}]
    \item \textbf{Description} \hfill \\
    The need to better produce, manage, and use energy is undeniable. In this course, we will discuss current energy situations and why we need energy conservation. We will explore why energy conservation alone is not enough and why we need alternative energy. Different types of alternative energy will be discussed: hydroelectric, geothermal, solar, wind, and biomass energy. Future technology will be briefly surveyed. We will studied successful and failure cases of alternative energy deployments. Policies related to alternative energy and energy conservation will be discussed. At the end of the class, students will combine all the knowledge gained throughout the semester to create an alternative energy or energy conservation project proposals for Phuket.
    \item \textbf{Recommended skills} \hfill \\
        Students are not required to have these skills. However, if students find themselves lacking any of these skills, it is highly recommended that appropriate study and/or training are sought for concurrently with the class.
        \begin{itemize}
            \item Although the class will be taught in Thai, ability to read texts in English is highly suggested. As the field of alternative energy is not well-developed in Thailand, most documents will be in English. For example, while a student doesn't need to completely understand this course syllabus, if a student can't grasp the main points of this document without consulting a dictionary, having a productive learning experience will be hard. 
            \item Being able to research for information and present it in a written format in a logical fashion are important in this class. If you have never written a research paper at all in high school, you should look up good research paper examples prior to having to submit one. Some good examples will be presented in the class.
        \end{itemize}
    \item \textbf{Organization} \hfill \\
    TODO TODO TODO
    \item \textbf{Course Objectives} \hfill \\
    \begin{enumerate}[label=\arabic*.]
        \item Students understand current energy demand situation and how the demand is being fulfilled.
        \item Students recognize the need for energy conservation and the overall effect it has in curbing world's energy demand.
        \item Students realize the need for alternative energy for world's development.
        \item Students can explain basic principles of current alternative energy technology: geothermal, bioenergy, wind energy, hydroelectric, solar energy. 
        \item Students are aware of future alternative energy technology and its current limitations.
        \item Students understand the effects on the community when an energy solution is implemented.
        \item Students can argue successful or failed alternative energy solutions.
        \item Students can evaluate the likelihood that an alternative energy solution will likely succeed or fail.
        \item Students can explain the advantages and disadvantages of different energy policies
        \item Combining all the knowledge learned throughout the course, students, in small groups, will develop energy solution proposals for Phuket.
    \end{enumerate}
    
    \item \textbf{Course Topics} \hfill \\
    \begin{enumerate}[label=\arabic*.]
        \item Current energy demand and current energy supply
        \item Future energy demand trend
        \item The importance of energy in development
        \item Conservation of energy: technology, techniques, and implementation
        \item 
        \item test2
        \item test3
    \end{enumerate}

\end{enumerate}


\end{document}



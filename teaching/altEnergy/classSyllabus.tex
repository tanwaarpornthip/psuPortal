\documentclass[a4paper,12pt]{article}

% Set the locale for linebreak to Thai
%\XeTeXlinebreaklocale "th"
% In English, when TeX tries to justify text,
% it will add some spaces between words
% For Thai, we "must not" add any space between words
% i.e. put "zero" space beteween words
%\XeTeXlinebreakskip = 0pt plus 0pt
% For a bit better(?) justified output
%\sloppy

% For any unicode characters, require XeTeX/XeLaTeX
\usepackage{fontspec}
\usepackage{subcaption}
\usepackage[document]{ragged2e}
\defaultfontfeatures{Mapping=tex-text} 

% Set main fonts
% For Thai, I recommend to scale the size to the uppercase size of latin alphabet
%\setmainfont[Scale=MatchUppercase,Mapping=tex-text]{TH Sarabun New}
\setmainfont{TeX Gyre Termes}				% Free Times

% Sans font
\setsansfont{TeX Gyre Heros}				% Free Helvetica

% Monospace font
\setmonofont{TeX Gyre Cursor}				% Free Courier
%%%%%%%%%%%%%%%%%%%%%%%%%%%%%%%%%%%%%%%%%%%%%%%%%%%%%%%%%%%%%%%%%%%%%%%%%
\usepackage[top=2cm, bottom=2cm, left=2cm, right=2cm]{geometry}
%\addto{\captionsthai}{\renewcommand{\abstractname}{รายละเอียดของโครงการโดยย่อ}}
\renewcommand{\abstractname}{Project Overview}

\usepackage{listings}
\lstset{
    language=TeX,
    basicstyle=\ttfamily,
    numbers=left,
    numberstyle=\small,
    breaklines=true,
    xleftmargin=.05\linewidth,
    frame=single,
    columns=fullflexible,
    captionpos=b,
    showstringspaces=false,
}



% Any percent sign marks a comment to the end of the line

% Every latex document starts with a documentclass declaration like this
% The option dvips allows for graphics, 12pt is the font size, and article
%   is the style

\usepackage{graphicx}
\usepackage{url}

% These are additional packages for "pdflatex", graphics, and to include
% hyperlinks inside a document.

%\setlength{\oddsidemargin}{0.25in}
%\setlength{\textwidth}{6.5in}
%\setlength{\topmargin}{0in}
%\setlength{\textheight}{8.5in}

\usepackage{authblk}
% Use thai separation for Thai document
%\renewcommand\Authand{ และ }

% IndentFirst line
\usepackage{indentfirst}

% Using hyperlink in document
\usepackage{hyperref}

% Using cleverref for referencing items in the document
\usepackage{cleveref}

% Wrap figure use
\usepackage{wrapfig}

% Table accommodation
%\usepackage{array}
\newcommand*{\thead}[1]{\textbf{#1}}

%Set unnumbered section
\setcounter{secnumdepth}{0}

%Paragraph indentation settings
\setlength{\RaggedRightParindent}{2em}

\begin{document}

% Everything after this becomes content
% Replace the text between curly brackets with your own

\title{Course syllabus\\ \large 988-342 Alternative Energy \& Energy conservation}
\author{Dr. Tanwa Arpornthip}
\date{}

% You can leave out "date" and it will be added automatically for today
% You can change the "\today" date to any text you like
% A blank date can be given if date is not desired in the title

\maketitle
% This command causes the title to be created in the document

\noindent\makebox[\linewidth]{\rule{\linewidth}{0.4pt}}
\begin{description}
\item [Office] Building 6, ESSAND faculty room
\item [Office Hours] \hfill \\
    Monday 14:00-14:50 ESSAND meeting room \hfill\\
    Friday 12:00-14:00 ESSAND meeting room
\item [Email address] \href{mailto:tanwa.a@phuket.psu.ac.th}{tanwa.a@phuket.psu.ac.th}
\item [Work phone] 076-276-435
\item [Web page] \href{http://www.essand.psu.ac.th/ESSAND/Tanwa}{http://www.essand.psu.ac.th/ESSAND/Tanwa}
\item [Course page] \href{http://www.essand.psu.ac.th/ESSAND/Tanwa/altEnergy}{http://www.essand.psu.ac.th/ESSAND/Tanwa/altEnergy}
\item [Class hours] Monday, Wednesday, Friday 16:30-17:30 Room 5407A
\end{description}
\noindent\makebox[\linewidth]{\rule{\linewidth}{0.4pt}}

\begin{enumerate}
    \item \textbf{Description} \hfill \\

    This course explores current technology in alternative energy and energy conservation.
    ;laskjf;lkj

    asdf;laskjdfas;lkfjl;askjf
\end{enumerate}


\end{document}



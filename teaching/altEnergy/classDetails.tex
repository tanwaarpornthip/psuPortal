\documentclass[a4paper,12pt]{article}

% Set the locale for linebreak to Thai
\XeTeXlinebreaklocale "th"
% In English, when TeX tries to justify text,
% it will add some spaces between words
% For Thai, we "must not" add any space between words
% i.e. put "zero" space beteween words
\XeTeXlinebreakskip = 0pt plus 0pt
% For a bit better(?) justified output
\sloppy

% For any unicode characters, require XeTeX/XeLaTeX
\usepackage{fontspec}
\defaultfontfeatures{Mapping=tex-text} 
\DeclareMathSizes{12}{14}{10}{10}


% Set main fonts
% For Thai, I recommend to scale the size to the uppercase size of latin alphabet
%\setmainfont[Scale=MatchUppercase,Mapping=tex-text]{TH Sarabun New}
\setmainfont{TeX Gyre Termes}				% Free Times

% Sans font
\setsansfont{TeX Gyre Heros}				% Free Helvetica

% Monospace font
\setmonofont{TeX Gyre Cursor}				% Free Courier

% Because latin font in Sarabun is Sans Serif, we prefer to use Serif font
\newfontfamily{\thaifont}[Scale=MatchUppercase,Mapping=tex-text]{TH Sarabun New}
\newfontfamily{\latinfont}[Mapping=tex-text]{TeX Gyre Termes}

% For automatic switching between languages
\usepackage[Latin,Thai]{ucharclasses}
% Set default font for Thai
\setTransitionTo{Thai}{\thaifont}
% Set default font for other characters
\setTransitionFrom{Thai}{\latinfont}

% Single spacing is too tight for Thai
\usepackage{setspace}
\onehalfspacing
% Set spacing in equation
\setlength{\jot}{10pt}

% Use hyperlink in document
\usepackage{hyperref}

% multirow multicolumn package
\usepackage{multirow}
\usepackage{multicol}
\usepackage{hhline}

% TabularX for the win
\usepackage{tabularx}
\usepackage{tabu}
\newcommand{\tblhdr}[1]{\multicolumn{1}{c|}{\textbf{#1}}}

% wrapfigure package
\usepackage{wrapfig}


% For thaialph numbering \thAlph
\usepackage{polyglossia}          
% Set the normal language to English
% i.e. numbering, latin characters will use English font
\setdefaultlanguage{english}
% When using Thai characters, the font will be automatically changed to Thai font
\setotherlanguage{thai}

% amsmath
\usepackage{mathtools}

%\AtBeginDocument\captionsthai               % Force the caption to Thai

% More references in gloss-thai of polyglossia package
% http://texdoc.net/texmf-dist/tex/latex/polyglossia/gloss-thai.ldf
% \thaiAlpha works well, the sequence is ก ข "ฃ" ค "ฅ" ฆ ง จ ...
% Normally, we need ก ข ค ง จ which is defined in \thaialph
% I'm not sure why it doesn't work. So, I just re-define it.
\def\thaialph#1{\expandafter\thalph\csname c@#1\endcsname}
\def\thalph#1{%
    \ifcase#1\or ก\or ข\or ค\or ง\or จ\or ฉ\or ช\or ซ\or
    ฌ\or ญ\or ฎ\or ฏ\or ฐ\or ฑ\or ฒ\or ณ\or ด\or ต\or ถ\or ท\or ธ\or น\or
    บ\or ป\or ผ\or ฝ\or พ\or ฟ\or ภ\or ม\or ย\or ร\or ฤ\or ล\or ฦ\or ว\or
    ศ\or ษ\or ส\or ห\or ฬ\or อ\else ฮ\else\xpg@ill@value{#1}{thalph}\fi}

% Again, re-define the sequence of Thai number.
\def\thainum#1{\expandafter\thainumber\csname c@#1\endcsname}
\def\thainumber#1{%
    \thaidigits{\number#1}%
}
\def\thaidigits#1{\expandafter\thdigits #1@ }
\def\thdigits#1{%
    \ifx @#1% then terminate
    \else
    \ifx0#1๐\else\ifx1#1๑\else\ifx2#1๒\else\ifx3#1๓\else\ifx4#1๔\else\ifx5#1๕\else\ifx6#1๖\else\ifx7#1๗\else\ifx8#1๘\else\ifx9#1๙\else#1\fi\fi\fi\fi\fi\fi\fi\fi\fi\fi
    \expandafter\thdigits
    \fi
}

%%%%%%%%%%%%%%%%%%%%%%%%%%%%%%%%%%%%%%%%%%%%%%%%%%%%%%%%%%%%%%%%%%%%%%%%%%
\usepackage[top=2cm, bottom=2cm, left=2cm, right=2cm]{geometry}

\usepackage{listings}
\lstset{
    language=TeX,
    basicstyle=\ttfamily,
    numbers=left,
    numberstyle=\small,
    breaklines=true,
    xleftmargin=.1\linewidth,
    frame=single,
    columns=fullflexible,
    captionpos=b,
    showstringspaces=false,
}

% Indent first line of paragraph
\usepackage{indentfirst}


% Indent lines in enumerate
\usepackage{enumitem}
\setenumerate{listparindent=\parindent}

%\renewcommand{\thesection}{\thainum{section}}

% Cleveref
\usepackage{cleveref}

% Add imageBank path
\graphicspath{{../../imageBank/}}


\begin{document}
\title{รายละเอียดกระบวนการจัดการเรียนการสอน\vspace{-2cm}}
\date{}
\maketitle

\begin{description}
    \item [\textbf{รายวิชา}] 988-342 พลังงานทางเลือกและการอนุรักษ์พลังงาน (Alternative energy and Energy conservaton)
    \item [\textbf{อาจารย์ผู้สอน}] ดร.ธันวา อาภรณ์ทิพย์
    \item [\textbf{คณะ}] เทคโนโลยีและสิ่งแวดล้อม
    \item [\textbf{สาขาวิชา}] สาขาวิชาเทคโนโลยีและการจัดการสิ่งแวดล้อม
\end{description}

\begin{enumerate}[label=\textbf{\arabic*},leftmargin=*]
    \item \noindent \textbf{หลักการและเหตุผล} \hfill \par
        ในปัจจุบันสังคมทั้งภายในประเทศและประชาคมโลกได้เริ่มตระหนักถึงการเปลี่ยนแปลงภูมิอากาศ (Climate change) และผลเสียในระยะยาวหากปัญหาไม่ได้รับการแก้ไข ซึ่งสามารถรับรู้ได้จากนโยบายต่าง ๆ ทั้งจากในประเทศ เช่น การประกาศลดก๊าซเรือนกระจก 20-25\% ภายในปี พ.ศ. 2573 หรือในระดับนานาชาติ เช่น หนังสือความตกลงปารีส (Paris Accord) ส่วนสำคัญของการเปลี่ยนแปลงภูมิอากาศคือปริมาณแก๊สเรือนกระจก (Greenhouse gas) ที่ได้รับการปลดปล่อยสู่ชั้นบรรยากาศจากการผลิต ใช้ และการบริหารจัดการพลังงาน ดังนั้นการเลือกใช้พลังงานทางเลือกและการอนุรักษ์พลังงานจะมีส่วนช่วยอย่างมากต่อการชะลอและแก้ปัญหาการเปลี่ยนแปลงภูมิอากาศ

        จากความสำคัญของการใช้พลังงานให้อย่างมีประสิทธิภาพ เพื่อลดปริมาณแก๊สเรือนกระจกในชั้นบรรยากาศ นักศึกษาของสาขาวิชาเทคโนโลยีและการจัดการสิ่งแวดล้อม ที่จะกลายเป็นกำลังสำคัญในการพัฒนาประชาคมโลก ย่อมจะหนีไม่พ้นการตัดสินใจที่ต้องเกี่ยวข้องกับพลังงานทางเลือกหรือการอนุรักษ์พลังงาน ไม่ว่านักศึกษาจะทำงานที่เกี่ยวข้องกับสาขาใด เนื่องจากรายวิชาพลังงานทางเลือกและการอนุรักษ์พลังงานยังไม่เคยได้รับการพัฒนามาก่อน ดังน้ันการพัฒนารายวิชาเพื่อช่วยให้นักศึกษามีความรู้ความเข้าใจมากขึ้นในหัวข้อดังกล่าว จึงมีความจำเป็นอย่างยิ่ง
    \item \noindent \textbf{วัตถุประสงค์ของรายวิชา} \hfill \par
    \begin{enumerate}[label=\arabic*.]
        \item นักเรียนมีความเข้าใจลักษณะความต้องการของพลังงานในปัจจุบัน และการตอบสนองต่อความต้องการทางพลังงาน
        \item นักเรียนทราบถึงความจำเป็นของการอนุรักษ์พลังงานและผลกระทบในภาพรวมของการลดความต้องการพลังงานของประชาคมโลก
        \item นักเรียนสามารถอธิบายการหลักการทำงานของเทคโนโลยีพลังงานทางเลือกที่มีอยู่ในปัจจุบัน โดยประกอบด้วย พลังงานความร้อนใต้พิภพ พลังงานชีวมวล พลังงานลม พลังงานน้ำ และพลังงานแสงอาทิตย์
        \item นักเรียนทราบถึงเทคโนโลยีพลังงานทางเลือกที่จะเกิดขึ้นในอนาคต รวมทั้งข้อจำกัดในปัจจุบัน
        \item นักเรียนตระหนักถึงผลกระทบต่อสังคมจากการเลือกใช้พลังงานทางเลือก
        \item นักเรียนสามารถวิเคราะห์เหตุผลที่โครงการพลังงานทางเลือกหนึ่ง ๆ จะสามารถประสบความสำเร็จหรือล้มเหลวได้
        \item นักเรียนสามารถประเมินโครงการพลังงานทางเลือกหนึ่ง ๆ ได้ว่ามีแนวโน้มที่จะประสบความสำเร็จหรือล้มเหลว
        \item นักเรียนสามารถอธิบายข้อดีและข้อเสียของนโยบายทางด้านพลังงานต่าง ๆ
        \item นักเรียนสามารถจับกลุ่มย่อยเพื่่อวิเคราะห์และนำเสนอโครงการพลังงานทางเลือกที่มีแนวโน้มใช้ได้จริงในจังหวัดภูเก็ต
    \end{enumerate}
    \item \noindent \textbf{จำนวนผู้เรียน} \hfill \par
        30
    \item \noindent \textbf{แนวคิด/ทฤษฎีที่ใช้ในการจัดการเรียนการสอน} \hfill \par
        \begin{enumerate}[label=\arabic*.]
            \item เพื่อส่งเสริมให้นักศึกษาได้เรียนรู้ทักษะที่จำเป็นต่อการทำงานในอนาคต และเรียนรู้การทำงานอย่างมืออาชีพ การจัดการเรียนการสอนในวิชานี้จึงเน้นนโยบายและการสร้างผลงานที่สะท้อนทักษะที่ันักศึกษาจำเป็นต้องใช้จริงในอนาคต รายวิชานี้จะเสริมสร้างให้นักศึกษาทำตัวเช่นเดียวกันกับที่ผู้เข้าทำงานจำเป็นต้องทำ ตัวอย่างนโยบายที่สะท้อนและเสริมสร้างความเป็นมืออาชีพ เช่น การไม่รับตรวจงานที่ส่งล่าช้าเกินกำหนดโดยเด็ดขาด (นอกจากกรณีที่มีเรื่องฉุกเฉินเกี่ยวกับครอบครัวหรือเจ็บป่วย) การไม่ยอมรับการคัดลอกผลงานโดยไม่มีการอ้างอิงโดยเด็ดขาด (Plagiarism) หรือการจัดให้มีการประเมินผลโดยเพื่อนร่วมงานในการทำงานกลุ่ม (Peer evaluation) ตัวอย่างผลงานและการบ้านที่เสริมสร้างความเป็นมืออาชีพ เช่น การสร้างข้อเสนอโครงการพลังงานทางเลือกที่สามารถใช้ได้จริง โดยรูปเล่มต้องจัดทำอย่างมีความสวยงามและประณีต การเสนอการวิเคราะห์ case study เพื่อหาข้อดีและข้อเสีย รวมถึงการที่รายวิชาไม่มีการจัดสอบหรือการบ้านที่ไม่มีผลกระทบกับงานในอนาคต เช่น ข้อสอบแบบปรนัย โดยความรู้ที่ต้องการทดสอบ สามารถประเมินได้ผ่านการแลกเปลี่ยนความคิดเห็นในห้องเรียน
            \item ในปัจจุบัน ความสามารถในการค้นหาข้อมูลเพิ่มเติม และวิจัยได้ด้วยตนเองเป็นสิ่งจำเป็นอย่างมากต่อการทำงานในยุคดิจิทัลที่ข้อมูลข่าวสารสามารถถูกตรวจสอบและอ้างอิงได้อย่างรวดเร็ว ดังนั้นผลงานส่วนใหญ่ของนักศึกษาที่ถูกสร้างในรายวิชานี้จะมาจากการศึกษาค้นคว้าข้อมูลโดยตัวนักศึกษาเอง โดยเอกสารส่วนหนึ่งจะมีการเตรียมโดยอาจารย์ผู้สอนเพื่อช่วยให้การศึกษาเป็นไปดังแนวทางที่ต้องการ
            \item ส่งเสริมให้นักศึกษามีการใช้การอ้างอิงผลงานที่ถูกหลัก เพื่อเตรียมความพร้อมของนักศึกษาต่อการทำงานในอนาคต
            \item รูปแบบการนำเสนอผลงาน จะมีการแบ่งนักศึกษาออกเป็นกลุ่มย่อย ๆ โดยให้แต่ละกลุ่มเป็นบริษัทที่นำเสนอโครงการทางด้านพลังงาน และมีการแข่งขันกันระหว่างกลุ่มเพื่อจำลองบรรยากาศการทำงานที่แท้จริง
            \item ไม่มีการจัดสอบโดยใช้ข้อสอบปรนัยหรืออัตนัย การวัดและประเมินผลทั้งหมดกระทำโดยผ่านการนำเสนอผลงาน การเขียนรายงาน และการนำเสนอโครงการตัวอย่างของนักศึกษาเท่านั้น ซึ่งการวัดผลในรูปแบบนี้ตรงกับทักษะและการวัดผลที่นักศึกษาจำต้องใช้ในอนาคต
        \end{enumerate}
    \item \textbf{แผนการสอน} \hfill \\
\begin{flushleft}
\begin{tabularx}{\textwidth}{|p{2cm}|p{4cm}|X|p{3cm}|}
\hline
\textbf{หัวข้อที่สอน}&
\textbf{Expected Learning Outcome}&\textbf{รายละเอียดกิจกรรมการเรียนการสอน}&\textbf{การประเมินผล} \\
\hline

%1st row%
ลักษณะความต้องการของพลังงานในปัจจุบัน และการตอบสนองต่อความต้องการทางพลังงาน &

1.นักศึกษาสามารถสรุปความสำคัญของการใช้พลังงานและความสัมพันธ์ของความต้องการพลังงานกับการพัฒนาประเทศไทยและสังคมโลก โดยการโยงความสัมพันธ์ระหว่างความต้องการทางด้านพลังงานกับการพัฒนาทางด้านเศรษฐกิจ เกษตรกรรม อุตสาหกรรม สังคม และวัฒนธรรมได้อย่างถูกต้อง \newline
2.นักศึกษาสามารถอธิบายความต้องการทางด้านพลังงานของประเทศไทยและสังคมโลกรวมถึงแนวโน้มความต้องการของพลังงานในอนาคต เป็นระยะเวลาอย่างน้อย 20 ปี โดยสามารถเรียงลำดับความต้องการของพลังงานตามประเภทของการใช้พลังงาน จากปริมาณมากที่สุดสู่น้อยที่สุดได้อย่างถูกต้อง \newline
3.นักศึกษาสามารถอธิบายการตอบสนองความต้องการทางด้านพลังงานของประเทศไทยและสังคมโลกรวมถึงแนวโน้มของการตอบสนองความต้องการของพลังงานในอนาคต เป็นระยะเวลาอย่างน้อย 20 ปี โดยสามารถเรียงลำดับความต้องการของการตอบสนอง ตามประเภทของแหล่งพลังงาน จากปริมาณมากที่สุดสู่น้อยที่สุดได้อย่างถูกต้อง \newline
2. asdfasdfasdf
&3 &4

\end{tabularx}
\end{flushleft}
\end{enumerate}

\end{document}

